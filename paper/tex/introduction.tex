\newcommand{\LineComment}[1]{\Statex \hfill\textit{#1}}
\vspace{-0.15in}
\section{Overview}
In proteomics, the composition of proteins is analyzed using mass spectrometry and the results are stored in plots called spectra.
Experimental spectra results are identified by identifying the theoretical spectra in FASTA databases that best matches the experimental composition.
A FASTA database can be up to (find the size).
When a protein sample is analyzed, a mass spectrometer can create up to 70 spectra per second (source?).
Tools such as Crux attempt to use database indices and multi-threading to speed up the analysis, however analyzing data is still slow (need numbers)\cite{crux}.\\
\newline
The proteomic community wants to share as many of these FASTA databases and spectra as possible to increase changes of finding important results.
Previous attempts at hosting servers to share and analyze this data have not scaled well as it is costly and requires too much maintenance.
Oftentimes, there is too much data and researchers must only examine a subset of the theoretical and / or experimental data.
This is bad, as researchers may miss important correlations.\\
\newline
Ideally, the community should not have to worry about scaling machine to meet their storage and computation needs.
Serverless computing offers a potential solution.
In serverless computing, a cloud provider provides the necessary resources for a task and the user specifies the code to execute the task.
% Serverless computing tasks are event driven and users are only changed when an event occurs.
This model benefits the user as users only need to pay for the resources used.
Splitting experimental and theoretical spectra across many computing instances would allow the proteomics community to check all potential matches in a timely manner.\\
\newline
In this paper, we present Ripple, a system that uses serverless computing to find the best theoretical spectra matches for experimental spectra.
The contributions of this paper are (1) we show one potential way to analyze data in a serverless computing setting, (2) we compare the different
serverless computing providers (Amazon Lambda, Microsoft Azure, and Google Functions) to determine the pros and cons of each provider, and (3) we
provide a way for the proteomics computing to analyze spectra without worrying about machine limitations.
%The model also benefits the cloud provider as they only need to provide the resources needed by the user, thus allowing them to save resouces.\\
%Proteins consists of chains of amino acids.
%A segment of the chain is called a peptide.
%The composition of peptides are analyzed using mass spectrometry and stored in plots call spectra.
%Each spectrum contains the m/z (charged mass) of pieces of the peptide correlating to the amino acid composition of the peptide.
%Comet is a piece of software that correlates these experimental spectra datasets with theoretical datasets generated using FASTA databases.
%The largest FASTA database is the database containing human proteins.
%This database contains 750 thousand entries.
%An entry in the database is a match if the protein contains the peptide.
%However proteins can have modifications.
%An n-modification means n of the amino acids in a protein differ from the base protein.\\
%\newline
%Manually scaling servers to handle all modifications is a challenge and usually researchers resort to only searching a subset of the database.
%This means the database may not contain the best match.
%Another problem in the proteomics community is researchers want an easy way for people to add new entries to a shared database.
%Serverless coputing offers a way for the proteomics community to cheaply and automatically scale these database and a potential way to share data in the community.
